%
% API Documentation for API Documentation
% Module cuon.Email2
%
% Generated by epydoc 3.0.1
% [Fri Feb 11 22:18:08 2011]
%

%%%%%%%%%%%%%%%%%%%%%%%%%%%%%%%%%%%%%%%%%%%%%%%%%%%%%%%%%%%%%%%%%%%%%%%%%%%
%%                          Module Description                           %%
%%%%%%%%%%%%%%%%%%%%%%%%%%%%%%%%%%%%%%%%%%%%%%%%%%%%%%%%%%%%%%%%%%%%%%%%%%%

    \index{cuon \textit{(package)}!cuon.Email2 \textit{(module)}|(}
\section{Module cuon.Email2}

    \label{cuon:Email2}
******************************************************************************
* Description:  SimpleMail ist ein Modul zum Versenden von Emails *
mit und ohne Anhaengen. * Filename:     simplemail.py * Created:      
2004-10-06 Gerold * License:      LGPL: 
http://www.opensource.org/licenses/lgpl-license.php * Requirements: Python 
2.3:  http://www.python.org * * Änderungen 2007 von Juergen Hamel, siehe 
Hinweise in der Datei * * Einfaches Beispiel: * *   from simplemail import 
Email *   Email( *       from\_address = "absender@domain.at", *       
to\_address = "empfaenger@domain.at", *       subject = "Betreff", *       
message = "Das ist der Nachrichtentext" *   ).send() * * 2004-03-13 Gerold 
*   - Umlaute in den Beschreibungen ausgebessert *   - Schreibweise der 
Kommentare wurde so umgesetzt dass auf einfache *     Art und Weise eine 
Uebersetzung stattfinden kann. *     Erklaerung: "de" steht fuer "deutsch" 
*                 "en" steht fuer "englisch" * * 2004-04-25 Gerold *   - 
Kleine Ausbesserungen in den Beschreibungen vorgenommen *   - Einfaches 
Beispiel in den Beschreibungstext integriert * * 2005-08-20 Gerold *   - 
Das Versenden von Anhaengen ermoeglicht * * 2005-09-28 Gerold *   - Die 
Rückgabe des Befehls "sendmail()" wird in das Attribut "statusdict" *     
der Instanz der Klasse "Email" geschrieben. So ist es jetzt auch moeglich, 
*     beim Versenden an mehrere Emailadressen, eine exakte Rueckmeldung 
ueber *     den Versandstatus zu erhalten. Das Format der Rueckgabe wird 
unter *     der Url 
http://www.python.org/doc/current/lib/SMTP-objects.html\#l2h-3493 *     
genau erklaert. * *     Hier ein Auszug aus dieser Erklaerung: *     This 
method will return normally if the mail is accepted for at least *     one 
recipient. Otherwise it will throw an exception. That is, if this *     
method does not throw an exception, then someone should get your mail. *
If this method does not throw an exception, it returns a dictionary, *     
with one entry for each recipient that was refused. Each entry contains *
a tuple of the SMTP error code and the accompanying error message sent *
by the server. * * 2005-09-29 Gerold *   - Das Format der Hilfe geaendert. 
*   - Ab jetzt wird auch der "User-Agent" im Header mitgesendet. *     
Jens, danke für die Idee. * * 2005-11-11 Gerold *   - Fixed: Das Versenden 
von Emails funktioniert jetzt auch wenn man *     sich am SMTP-Server mit 
Benutzername und Passwort anmelden muss. *     ChrisSek, danke für den 
Hinweis. *   - Es war, glaube ich, recht lästig, dass Testemails gesendet 
wurden, *     wenn man dieses Modul ausführte. Ich habe es so geändert, 
dass die *     Testemails nur mehr dann gesendet werden, wenn man dieses 
Modul mit *     dem Parameter "test" aufruft. Z.B. ``python simplemail.py 
test`` * * 2005-12-10 Gerold *   - Schlampigkeitsfehler ausgebessert. Es 
wurde ein Fehler gemeldet, wenn *     man beim Initialisieren der Klasse 
Email auch den Dateinamen eines *     Attachments übergeben hatte. Es war 
ein Unterstrich zu viel, der *     Entfernt wurde. * * 2006-03-22 Gerold *
- Klassen fuer CC-Empfaenger und BCC-Empfaenger hinzugefuegt. *     Ab 
jetzt können Emails auch an CC und BCC gesendet werden. *     Wie das 
funktioniert sieht man in der Funktion ``testen()`` * * 2006-03-30 Gerold *
- Reply-to (Antwort an) kann jetzt auch angegeben werden. * * 2006-05-28 
Gerold *   - Wortlaut des Headers "User-Agent" geändert. *   - Da nicht 
jeder SMTP-Server das Datum automatisch zum Header hinzufügt, *     wird ab
jetzt das Datum beim Senden hinzugefügt. *     (Karl, danke für den 
wichtigen Hinweis.) * 2005-06-08 Gerold *   - Fehlerklassen von 
**SimpleMail\_Exception** abgeleitet. Damit wird *     bei einem Fehler 
jetzt auch eine aussagekräftigere Fehlermeldung *     ausgegeben. Dabei 
habe ich auch die vertauschten Fehlermeldungen *     ausgetauscht. 
(Rebecca, danke für die Meldung.) *   - Da die Klassen **CCRecipients** und
**BCCRecipients** sowiso von *     **Recipients** abgeleitet wurden, kann 
ich mir das Überschreiben *     der Initialisierung (\_\_init\_\_) und die 
Angabe der Slots sparen. * 
******************************************************************************


%%%%%%%%%%%%%%%%%%%%%%%%%%%%%%%%%%%%%%%%%%%%%%%%%%%%%%%%%%%%%%%%%%%%%%%%%%%
%%                               Variables                               %%
%%%%%%%%%%%%%%%%%%%%%%%%%%%%%%%%%%%%%%%%%%%%%%%%%%%%%%%%%%%%%%%%%%%%%%%%%%%

  \subsection{Variables}

    \vspace{-1cm}
\hspace{\varindent}\begin{longtable}{|p{\varnamewidth}|p{\vardescrwidth}|l}
\cline{1-2}
\cline{1-2} \centering \textbf{Name} & \centering \textbf{Description}& \\
\cline{1-2}
\endhead\cline{1-2}\multicolumn{3}{r}{\small\textit{continued on next page}}\\\endfoot\cline{1-2}
\endlastfoot\raggedright \_\-\_\-p\-a\-c\-k\-a\-g\-e\-\_\-\_\- & \raggedright \textbf{Value:} 
{\tt \texttt{'}\texttt{cuon}\texttt{'}}&\\
\cline{1-2}
\end{longtable}


%%%%%%%%%%%%%%%%%%%%%%%%%%%%%%%%%%%%%%%%%%%%%%%%%%%%%%%%%%%%%%%%%%%%%%%%%%%
%%                           Class Description                           %%
%%%%%%%%%%%%%%%%%%%%%%%%%%%%%%%%%%%%%%%%%%%%%%%%%%%%%%%%%%%%%%%%%%%%%%%%%%%

    \index{cuon \textit{(package)}!cuon.Email2 \textit{(module)}!cuon.Email2.SimpleMail\_Exception \textit{(class)}|(}
\subsection{Class SimpleMail\_Exception}

    \label{cuon:Email2:SimpleMail_Exception}
\begin{tabular}{cccccccccc}
% Line for object, linespec=[False, False, False]
\multicolumn{2}{r}{\settowidth{\BCL}{object}\multirow{2}{\BCL}{object}}
&&
&&
&&
  \\\cline{3-3}
  &&\multicolumn{1}{c|}{}
&&
&&
&&
  \\
% Line for exceptions.BaseException, linespec=[False, False]
\multicolumn{4}{r}{\settowidth{\BCL}{exceptions.BaseException}\multirow{2}{\BCL}{exceptions.BaseException}}
&&
&&
  \\\cline{5-5}
  &&&&\multicolumn{1}{c|}{}
&&
&&
  \\
% Line for exceptions.Exception, linespec=[False]
\multicolumn{6}{r}{\settowidth{\BCL}{exceptions.Exception}\multirow{2}{\BCL}{exceptions.Exception}}
&&
  \\\cline{7-7}
  &&&&&&\multicolumn{1}{c|}{}
&&
  \\
&&&&&&\multicolumn{2}{l}{\textbf{cuon.Email2.SimpleMail\_Exception}}
\end{tabular}

\textbf{Known Subclasses:}
cuon.Email2.AttachmentNotFound\_Exception,
    cuon.Email2.NoFromAddress\_Exception,
    cuon.Email2.NoSubject\_Exception,
    cuon.Email2.NoToAddress\_Exception

SimpleMail Base-Exception


%%%%%%%%%%%%%%%%%%%%%%%%%%%%%%%%%%%%%%%%%%%%%%%%%%%%%%%%%%%%%%%%%%%%%%%%%%%
%%                                Methods                                %%
%%%%%%%%%%%%%%%%%%%%%%%%%%%%%%%%%%%%%%%%%%%%%%%%%%%%%%%%%%%%%%%%%%%%%%%%%%%

  \subsubsection{Methods}

    \vspace{0.5ex}

\hspace{.8\funcindent}\begin{boxedminipage}{\funcwidth}

    \raggedright \textbf{\_\_str\_\_}(\textit{self})

\setlength{\parskip}{2ex}
    str(x)

\setlength{\parskip}{1ex}
      Overrides: object.\_\_str\_\_ 	extit{(inherited documentation)}

    \end{boxedminipage}


\large{\textbf{\textit{Inherited from exceptions.Exception}}}

\begin{quote}
\_\_init\_\_(), \_\_new\_\_()
\end{quote}

\large{\textbf{\textit{Inherited from exceptions.BaseException}}}

\begin{quote}
\_\_delattr\_\_(), \_\_getattribute\_\_(), \_\_getitem\_\_(), \_\_getslice\_\_(), \_\_reduce\_\_(), \_\_repr\_\_(), \_\_setattr\_\_(), \_\_setstate\_\_(), \_\_unicode\_\_()
\end{quote}

\large{\textbf{\textit{Inherited from object}}}

\begin{quote}
\_\_format\_\_(), \_\_hash\_\_(), \_\_reduce\_ex\_\_(), \_\_sizeof\_\_(), \_\_subclasshook\_\_()
\end{quote}

%%%%%%%%%%%%%%%%%%%%%%%%%%%%%%%%%%%%%%%%%%%%%%%%%%%%%%%%%%%%%%%%%%%%%%%%%%%
%%                              Properties                               %%
%%%%%%%%%%%%%%%%%%%%%%%%%%%%%%%%%%%%%%%%%%%%%%%%%%%%%%%%%%%%%%%%%%%%%%%%%%%

  \subsubsection{Properties}

    \vspace{-1cm}
\hspace{\varindent}\begin{longtable}{|p{\varnamewidth}|p{\vardescrwidth}|l}
\cline{1-2}
\cline{1-2} \centering \textbf{Name} & \centering \textbf{Description}& \\
\cline{1-2}
\endhead\cline{1-2}\multicolumn{3}{r}{\small\textit{continued on next page}}\\\endfoot\cline{1-2}
\endlastfoot\multicolumn{2}{|l|}{\textit{Inherited from exceptions.BaseException}}\\
\multicolumn{2}{|p{\varwidth}|}{\raggedright args, message}\\
\cline{1-2}
\multicolumn{2}{|l|}{\textit{Inherited from object}}\\
\multicolumn{2}{|p{\varwidth}|}{\raggedright \_\_class\_\_}\\
\cline{1-2}
\end{longtable}

    \index{cuon \textit{(package)}!cuon.Email2 \textit{(module)}!cuon.Email2.SimpleMail\_Exception \textit{(class)}|)}

%%%%%%%%%%%%%%%%%%%%%%%%%%%%%%%%%%%%%%%%%%%%%%%%%%%%%%%%%%%%%%%%%%%%%%%%%%%
%%                           Class Description                           %%
%%%%%%%%%%%%%%%%%%%%%%%%%%%%%%%%%%%%%%%%%%%%%%%%%%%%%%%%%%%%%%%%%%%%%%%%%%%

    \index{cuon \textit{(package)}!cuon.Email2 \textit{(module)}!cuon.Email2.NoFromAddress\_Exception \textit{(class)}|(}
\subsection{Class NoFromAddress\_Exception}

    \label{cuon:Email2:NoFromAddress_Exception}
\begin{tabular}{cccccccccccc}
% Line for object, linespec=[False, False, False, False]
\multicolumn{2}{r}{\settowidth{\BCL}{object}\multirow{2}{\BCL}{object}}
&&
&&
&&
&&
  \\\cline{3-3}
  &&\multicolumn{1}{c|}{}
&&
&&
&&
&&
  \\
% Line for exceptions.BaseException, linespec=[False, False, False]
\multicolumn{4}{r}{\settowidth{\BCL}{exceptions.BaseException}\multirow{2}{\BCL}{exceptions.BaseException}}
&&
&&
&&
  \\\cline{5-5}
  &&&&\multicolumn{1}{c|}{}
&&
&&
&&
  \\
% Line for exceptions.Exception, linespec=[False, False]
\multicolumn{6}{r}{\settowidth{\BCL}{exceptions.Exception}\multirow{2}{\BCL}{exceptions.Exception}}
&&
&&
  \\\cline{7-7}
  &&&&&&\multicolumn{1}{c|}{}
&&
&&
  \\
% Line for cuon.Email2.SimpleMail\_Exception, linespec=[False]
\multicolumn{8}{r}{\settowidth{\BCL}{cuon.Email2.SimpleMail\_Exception}\multirow{2}{\BCL}{cuon.Email2.SimpleMail\_Exception}}
&&
  \\\cline{9-9}
  &&&&&&&&\multicolumn{1}{c|}{}
&&
  \\
&&&&&&&&\multicolumn{2}{l}{\textbf{cuon.Email2.NoFromAddress\_Exception}}
\end{tabular}

en: No sender address de: Es wurde keine Absenderadresse angegeben


%%%%%%%%%%%%%%%%%%%%%%%%%%%%%%%%%%%%%%%%%%%%%%%%%%%%%%%%%%%%%%%%%%%%%%%%%%%
%%                                Methods                                %%
%%%%%%%%%%%%%%%%%%%%%%%%%%%%%%%%%%%%%%%%%%%%%%%%%%%%%%%%%%%%%%%%%%%%%%%%%%%

  \subsubsection{Methods}


\large{\textbf{\textit{Inherited from cuon.Email2.SimpleMail\_Exception\textit{(Section \ref{cuon:Email2:SimpleMail_Exception})}}}}

\begin{quote}
\_\_str\_\_()
\end{quote}

\large{\textbf{\textit{Inherited from exceptions.Exception}}}

\begin{quote}
\_\_init\_\_(), \_\_new\_\_()
\end{quote}

\large{\textbf{\textit{Inherited from exceptions.BaseException}}}

\begin{quote}
\_\_delattr\_\_(), \_\_getattribute\_\_(), \_\_getitem\_\_(), \_\_getslice\_\_(), \_\_reduce\_\_(), \_\_repr\_\_(), \_\_setattr\_\_(), \_\_setstate\_\_(), \_\_unicode\_\_()
\end{quote}

\large{\textbf{\textit{Inherited from object}}}

\begin{quote}
\_\_format\_\_(), \_\_hash\_\_(), \_\_reduce\_ex\_\_(), \_\_sizeof\_\_(), \_\_subclasshook\_\_()
\end{quote}

%%%%%%%%%%%%%%%%%%%%%%%%%%%%%%%%%%%%%%%%%%%%%%%%%%%%%%%%%%%%%%%%%%%%%%%%%%%
%%                              Properties                               %%
%%%%%%%%%%%%%%%%%%%%%%%%%%%%%%%%%%%%%%%%%%%%%%%%%%%%%%%%%%%%%%%%%%%%%%%%%%%

  \subsubsection{Properties}

    \vspace{-1cm}
\hspace{\varindent}\begin{longtable}{|p{\varnamewidth}|p{\vardescrwidth}|l}
\cline{1-2}
\cline{1-2} \centering \textbf{Name} & \centering \textbf{Description}& \\
\cline{1-2}
\endhead\cline{1-2}\multicolumn{3}{r}{\small\textit{continued on next page}}\\\endfoot\cline{1-2}
\endlastfoot\multicolumn{2}{|l|}{\textit{Inherited from exceptions.BaseException}}\\
\multicolumn{2}{|p{\varwidth}|}{\raggedright args, message}\\
\cline{1-2}
\multicolumn{2}{|l|}{\textit{Inherited from object}}\\
\multicolumn{2}{|p{\varwidth}|}{\raggedright \_\_class\_\_}\\
\cline{1-2}
\end{longtable}

    \index{cuon \textit{(package)}!cuon.Email2 \textit{(module)}!cuon.Email2.NoFromAddress\_Exception \textit{(class)}|)}

%%%%%%%%%%%%%%%%%%%%%%%%%%%%%%%%%%%%%%%%%%%%%%%%%%%%%%%%%%%%%%%%%%%%%%%%%%%
%%                           Class Description                           %%
%%%%%%%%%%%%%%%%%%%%%%%%%%%%%%%%%%%%%%%%%%%%%%%%%%%%%%%%%%%%%%%%%%%%%%%%%%%

    \index{cuon \textit{(package)}!cuon.Email2 \textit{(module)}!cuon.Email2.NoToAddress\_Exception \textit{(class)}|(}
\subsection{Class NoToAddress\_Exception}

    \label{cuon:Email2:NoToAddress_Exception}
\begin{tabular}{cccccccccccc}
% Line for object, linespec=[False, False, False, False]
\multicolumn{2}{r}{\settowidth{\BCL}{object}\multirow{2}{\BCL}{object}}
&&
&&
&&
&&
  \\\cline{3-3}
  &&\multicolumn{1}{c|}{}
&&
&&
&&
&&
  \\
% Line for exceptions.BaseException, linespec=[False, False, False]
\multicolumn{4}{r}{\settowidth{\BCL}{exceptions.BaseException}\multirow{2}{\BCL}{exceptions.BaseException}}
&&
&&
&&
  \\\cline{5-5}
  &&&&\multicolumn{1}{c|}{}
&&
&&
&&
  \\
% Line for exceptions.Exception, linespec=[False, False]
\multicolumn{6}{r}{\settowidth{\BCL}{exceptions.Exception}\multirow{2}{\BCL}{exceptions.Exception}}
&&
&&
  \\\cline{7-7}
  &&&&&&\multicolumn{1}{c|}{}
&&
&&
  \\
% Line for cuon.Email2.SimpleMail\_Exception, linespec=[False]
\multicolumn{8}{r}{\settowidth{\BCL}{cuon.Email2.SimpleMail\_Exception}\multirow{2}{\BCL}{cuon.Email2.SimpleMail\_Exception}}
&&
  \\\cline{9-9}
  &&&&&&&&\multicolumn{1}{c|}{}
&&
  \\
&&&&&&&&\multicolumn{2}{l}{\textbf{cuon.Email2.NoToAddress\_Exception}}
\end{tabular}

en: No recipient address de: Es wurde keine Empfaengeradresse angegeben


%%%%%%%%%%%%%%%%%%%%%%%%%%%%%%%%%%%%%%%%%%%%%%%%%%%%%%%%%%%%%%%%%%%%%%%%%%%
%%                                Methods                                %%
%%%%%%%%%%%%%%%%%%%%%%%%%%%%%%%%%%%%%%%%%%%%%%%%%%%%%%%%%%%%%%%%%%%%%%%%%%%

  \subsubsection{Methods}


\large{\textbf{\textit{Inherited from cuon.Email2.SimpleMail\_Exception\textit{(Section \ref{cuon:Email2:SimpleMail_Exception})}}}}

\begin{quote}
\_\_str\_\_()
\end{quote}

\large{\textbf{\textit{Inherited from exceptions.Exception}}}

\begin{quote}
\_\_init\_\_(), \_\_new\_\_()
\end{quote}

\large{\textbf{\textit{Inherited from exceptions.BaseException}}}

\begin{quote}
\_\_delattr\_\_(), \_\_getattribute\_\_(), \_\_getitem\_\_(), \_\_getslice\_\_(), \_\_reduce\_\_(), \_\_repr\_\_(), \_\_setattr\_\_(), \_\_setstate\_\_(), \_\_unicode\_\_()
\end{quote}

\large{\textbf{\textit{Inherited from object}}}

\begin{quote}
\_\_format\_\_(), \_\_hash\_\_(), \_\_reduce\_ex\_\_(), \_\_sizeof\_\_(), \_\_subclasshook\_\_()
\end{quote}

%%%%%%%%%%%%%%%%%%%%%%%%%%%%%%%%%%%%%%%%%%%%%%%%%%%%%%%%%%%%%%%%%%%%%%%%%%%
%%                              Properties                               %%
%%%%%%%%%%%%%%%%%%%%%%%%%%%%%%%%%%%%%%%%%%%%%%%%%%%%%%%%%%%%%%%%%%%%%%%%%%%

  \subsubsection{Properties}

    \vspace{-1cm}
\hspace{\varindent}\begin{longtable}{|p{\varnamewidth}|p{\vardescrwidth}|l}
\cline{1-2}
\cline{1-2} \centering \textbf{Name} & \centering \textbf{Description}& \\
\cline{1-2}
\endhead\cline{1-2}\multicolumn{3}{r}{\small\textit{continued on next page}}\\\endfoot\cline{1-2}
\endlastfoot\multicolumn{2}{|l|}{\textit{Inherited from exceptions.BaseException}}\\
\multicolumn{2}{|p{\varwidth}|}{\raggedright args, message}\\
\cline{1-2}
\multicolumn{2}{|l|}{\textit{Inherited from object}}\\
\multicolumn{2}{|p{\varwidth}|}{\raggedright \_\_class\_\_}\\
\cline{1-2}
\end{longtable}

    \index{cuon \textit{(package)}!cuon.Email2 \textit{(module)}!cuon.Email2.NoToAddress\_Exception \textit{(class)}|)}

%%%%%%%%%%%%%%%%%%%%%%%%%%%%%%%%%%%%%%%%%%%%%%%%%%%%%%%%%%%%%%%%%%%%%%%%%%%
%%                           Class Description                           %%
%%%%%%%%%%%%%%%%%%%%%%%%%%%%%%%%%%%%%%%%%%%%%%%%%%%%%%%%%%%%%%%%%%%%%%%%%%%

    \index{cuon \textit{(package)}!cuon.Email2 \textit{(module)}!cuon.Email2.NoSubject\_Exception \textit{(class)}|(}
\subsection{Class NoSubject\_Exception}

    \label{cuon:Email2:NoSubject_Exception}
\begin{tabular}{cccccccccccc}
% Line for object, linespec=[False, False, False, False]
\multicolumn{2}{r}{\settowidth{\BCL}{object}\multirow{2}{\BCL}{object}}
&&
&&
&&
&&
  \\\cline{3-3}
  &&\multicolumn{1}{c|}{}
&&
&&
&&
&&
  \\
% Line for exceptions.BaseException, linespec=[False, False, False]
\multicolumn{4}{r}{\settowidth{\BCL}{exceptions.BaseException}\multirow{2}{\BCL}{exceptions.BaseException}}
&&
&&
&&
  \\\cline{5-5}
  &&&&\multicolumn{1}{c|}{}
&&
&&
&&
  \\
% Line for exceptions.Exception, linespec=[False, False]
\multicolumn{6}{r}{\settowidth{\BCL}{exceptions.Exception}\multirow{2}{\BCL}{exceptions.Exception}}
&&
&&
  \\\cline{7-7}
  &&&&&&\multicolumn{1}{c|}{}
&&
&&
  \\
% Line for cuon.Email2.SimpleMail\_Exception, linespec=[False]
\multicolumn{8}{r}{\settowidth{\BCL}{cuon.Email2.SimpleMail\_Exception}\multirow{2}{\BCL}{cuon.Email2.SimpleMail\_Exception}}
&&
  \\\cline{9-9}
  &&&&&&&&\multicolumn{1}{c|}{}
&&
  \\
&&&&&&&&\multicolumn{2}{l}{\textbf{cuon.Email2.NoSubject\_Exception}}
\end{tabular}

en: No subject de: Es wurde kein Betreff angegeben


%%%%%%%%%%%%%%%%%%%%%%%%%%%%%%%%%%%%%%%%%%%%%%%%%%%%%%%%%%%%%%%%%%%%%%%%%%%
%%                                Methods                                %%
%%%%%%%%%%%%%%%%%%%%%%%%%%%%%%%%%%%%%%%%%%%%%%%%%%%%%%%%%%%%%%%%%%%%%%%%%%%

  \subsubsection{Methods}


\large{\textbf{\textit{Inherited from cuon.Email2.SimpleMail\_Exception\textit{(Section \ref{cuon:Email2:SimpleMail_Exception})}}}}

\begin{quote}
\_\_str\_\_()
\end{quote}

\large{\textbf{\textit{Inherited from exceptions.Exception}}}

\begin{quote}
\_\_init\_\_(), \_\_new\_\_()
\end{quote}

\large{\textbf{\textit{Inherited from exceptions.BaseException}}}

\begin{quote}
\_\_delattr\_\_(), \_\_getattribute\_\_(), \_\_getitem\_\_(), \_\_getslice\_\_(), \_\_reduce\_\_(), \_\_repr\_\_(), \_\_setattr\_\_(), \_\_setstate\_\_(), \_\_unicode\_\_()
\end{quote}

\large{\textbf{\textit{Inherited from object}}}

\begin{quote}
\_\_format\_\_(), \_\_hash\_\_(), \_\_reduce\_ex\_\_(), \_\_sizeof\_\_(), \_\_subclasshook\_\_()
\end{quote}

%%%%%%%%%%%%%%%%%%%%%%%%%%%%%%%%%%%%%%%%%%%%%%%%%%%%%%%%%%%%%%%%%%%%%%%%%%%
%%                              Properties                               %%
%%%%%%%%%%%%%%%%%%%%%%%%%%%%%%%%%%%%%%%%%%%%%%%%%%%%%%%%%%%%%%%%%%%%%%%%%%%

  \subsubsection{Properties}

    \vspace{-1cm}
\hspace{\varindent}\begin{longtable}{|p{\varnamewidth}|p{\vardescrwidth}|l}
\cline{1-2}
\cline{1-2} \centering \textbf{Name} & \centering \textbf{Description}& \\
\cline{1-2}
\endhead\cline{1-2}\multicolumn{3}{r}{\small\textit{continued on next page}}\\\endfoot\cline{1-2}
\endlastfoot\multicolumn{2}{|l|}{\textit{Inherited from exceptions.BaseException}}\\
\multicolumn{2}{|p{\varwidth}|}{\raggedright args, message}\\
\cline{1-2}
\multicolumn{2}{|l|}{\textit{Inherited from object}}\\
\multicolumn{2}{|p{\varwidth}|}{\raggedright \_\_class\_\_}\\
\cline{1-2}
\end{longtable}

    \index{cuon \textit{(package)}!cuon.Email2 \textit{(module)}!cuon.Email2.NoSubject\_Exception \textit{(class)}|)}

%%%%%%%%%%%%%%%%%%%%%%%%%%%%%%%%%%%%%%%%%%%%%%%%%%%%%%%%%%%%%%%%%%%%%%%%%%%
%%                           Class Description                           %%
%%%%%%%%%%%%%%%%%%%%%%%%%%%%%%%%%%%%%%%%%%%%%%%%%%%%%%%%%%%%%%%%%%%%%%%%%%%

    \index{cuon \textit{(package)}!cuon.Email2 \textit{(module)}!cuon.Email2.AttachmentNotFound\_Exception \textit{(class)}|(}
\subsection{Class AttachmentNotFound\_Exception}

    \label{cuon:Email2:AttachmentNotFound_Exception}
\begin{tabular}{cccccccccccc}
% Line for object, linespec=[False, False, False, False]
\multicolumn{2}{r}{\settowidth{\BCL}{object}\multirow{2}{\BCL}{object}}
&&
&&
&&
&&
  \\\cline{3-3}
  &&\multicolumn{1}{c|}{}
&&
&&
&&
&&
  \\
% Line for exceptions.BaseException, linespec=[False, False, False]
\multicolumn{4}{r}{\settowidth{\BCL}{exceptions.BaseException}\multirow{2}{\BCL}{exceptions.BaseException}}
&&
&&
&&
  \\\cline{5-5}
  &&&&\multicolumn{1}{c|}{}
&&
&&
&&
  \\
% Line for exceptions.Exception, linespec=[False, False]
\multicolumn{6}{r}{\settowidth{\BCL}{exceptions.Exception}\multirow{2}{\BCL}{exceptions.Exception}}
&&
&&
  \\\cline{7-7}
  &&&&&&\multicolumn{1}{c|}{}
&&
&&
  \\
% Line for cuon.Email2.SimpleMail\_Exception, linespec=[False]
\multicolumn{8}{r}{\settowidth{\BCL}{cuon.Email2.SimpleMail\_Exception}\multirow{2}{\BCL}{cuon.Email2.SimpleMail\_Exception}}
&&
  \\\cline{9-9}
  &&&&&&&&\multicolumn{1}{c|}{}
&&
  \\
&&&&&&&&\multicolumn{2}{l}{\textbf{cuon.Email2.AttachmentNotFound\_Exception}}
\end{tabular}

en: Attachment not found de: Das uebergebene Attachment wurde nicht 
gefunden


%%%%%%%%%%%%%%%%%%%%%%%%%%%%%%%%%%%%%%%%%%%%%%%%%%%%%%%%%%%%%%%%%%%%%%%%%%%
%%                                Methods                                %%
%%%%%%%%%%%%%%%%%%%%%%%%%%%%%%%%%%%%%%%%%%%%%%%%%%%%%%%%%%%%%%%%%%%%%%%%%%%

  \subsubsection{Methods}


\large{\textbf{\textit{Inherited from cuon.Email2.SimpleMail\_Exception\textit{(Section \ref{cuon:Email2:SimpleMail_Exception})}}}}

\begin{quote}
\_\_str\_\_()
\end{quote}

\large{\textbf{\textit{Inherited from exceptions.Exception}}}

\begin{quote}
\_\_init\_\_(), \_\_new\_\_()
\end{quote}

\large{\textbf{\textit{Inherited from exceptions.BaseException}}}

\begin{quote}
\_\_delattr\_\_(), \_\_getattribute\_\_(), \_\_getitem\_\_(), \_\_getslice\_\_(), \_\_reduce\_\_(), \_\_repr\_\_(), \_\_setattr\_\_(), \_\_setstate\_\_(), \_\_unicode\_\_()
\end{quote}

\large{\textbf{\textit{Inherited from object}}}

\begin{quote}
\_\_format\_\_(), \_\_hash\_\_(), \_\_reduce\_ex\_\_(), \_\_sizeof\_\_(), \_\_subclasshook\_\_()
\end{quote}

%%%%%%%%%%%%%%%%%%%%%%%%%%%%%%%%%%%%%%%%%%%%%%%%%%%%%%%%%%%%%%%%%%%%%%%%%%%
%%                              Properties                               %%
%%%%%%%%%%%%%%%%%%%%%%%%%%%%%%%%%%%%%%%%%%%%%%%%%%%%%%%%%%%%%%%%%%%%%%%%%%%

  \subsubsection{Properties}

    \vspace{-1cm}
\hspace{\varindent}\begin{longtable}{|p{\varnamewidth}|p{\vardescrwidth}|l}
\cline{1-2}
\cline{1-2} \centering \textbf{Name} & \centering \textbf{Description}& \\
\cline{1-2}
\endhead\cline{1-2}\multicolumn{3}{r}{\small\textit{continued on next page}}\\\endfoot\cline{1-2}
\endlastfoot\multicolumn{2}{|l|}{\textit{Inherited from exceptions.BaseException}}\\
\multicolumn{2}{|p{\varwidth}|}{\raggedright args, message}\\
\cline{1-2}
\multicolumn{2}{|l|}{\textit{Inherited from object}}\\
\multicolumn{2}{|p{\varwidth}|}{\raggedright \_\_class\_\_}\\
\cline{1-2}
\end{longtable}

    \index{cuon \textit{(package)}!cuon.Email2 \textit{(module)}!cuon.Email2.AttachmentNotFound\_Exception \textit{(class)}|)}

%%%%%%%%%%%%%%%%%%%%%%%%%%%%%%%%%%%%%%%%%%%%%%%%%%%%%%%%%%%%%%%%%%%%%%%%%%%
%%                           Class Description                           %%
%%%%%%%%%%%%%%%%%%%%%%%%%%%%%%%%%%%%%%%%%%%%%%%%%%%%%%%%%%%%%%%%%%%%%%%%%%%

    \index{cuon \textit{(package)}!cuon.Email2 \textit{(module)}!cuon.Email2.Attachments \textit{(class)}|(}
\subsection{Class Attachments}

    \label{cuon:Email2:Attachments}
\begin{tabular}{cccccc}
% Line for object, linespec=[False]
\multicolumn{2}{r}{\settowidth{\BCL}{object}\multirow{2}{\BCL}{object}}
&&
  \\\cline{3-3}
  &&\multicolumn{1}{c|}{}
&&
  \\
&&\multicolumn{2}{l}{\textbf{cuon.Email2.Attachments}}
\end{tabular}

de: Dieses Objekt stellt die Anhaenge einer Email dar


%%%%%%%%%%%%%%%%%%%%%%%%%%%%%%%%%%%%%%%%%%%%%%%%%%%%%%%%%%%%%%%%%%%%%%%%%%%
%%                                Methods                                %%
%%%%%%%%%%%%%%%%%%%%%%%%%%%%%%%%%%%%%%%%%%%%%%%%%%%%%%%%%%%%%%%%%%%%%%%%%%%

  \subsubsection{Methods}

    \vspace{0.5ex}

\hspace{.8\funcindent}\begin{boxedminipage}{\funcwidth}

    \raggedright \textbf{\_\_init\_\_}(\textit{self})

    \vspace{-1.5ex}

    \rule{\textwidth}{0.5\fboxrule}
\setlength{\parskip}{2ex}
    de: Initialisiert die Anhaenge

\setlength{\parskip}{1ex}
      Overrides: object.\_\_init\_\_

    \end{boxedminipage}

    \label{cuon:Email2:Attachments:add_filename}
    \index{cuon \textit{(package)}!cuon.Email2 \textit{(module)}!cuon.Email2.Attachments \textit{(class)}!cuon.Email2.Attachments.add\_filename \textit{(method)}}

    \vspace{0.5ex}

\hspace{.8\funcindent}\begin{boxedminipage}{\funcwidth}

    \raggedright \textbf{add\_filename}(\textit{self}, \textit{dicFile}={\tt \texttt{'}\texttt{}\texttt{'}})

    \vspace{-1.5ex}

    \rule{\textwidth}{0.5\fboxrule}
\setlength{\parskip}{2ex}
    en: Adds an attachment de: Fuegt einen neuen Anhang hinzu

\setlength{\parskip}{1ex}
    \end{boxedminipage}

    \label{cuon:Email2:Attachments:count}
    \index{cuon \textit{(package)}!cuon.Email2 \textit{(module)}!cuon.Email2.Attachments \textit{(class)}!cuon.Email2.Attachments.count \textit{(method)}}

    \vspace{0.5ex}

\hspace{.8\funcindent}\begin{boxedminipage}{\funcwidth}

    \raggedright \textbf{count}(\textit{self})

    \vspace{-1.5ex}

    \rule{\textwidth}{0.5\fboxrule}
\setlength{\parskip}{2ex}
    en: Returns the number of attachments de: Gibt die Anzahl der Anhaenge 
    zurueck

\setlength{\parskip}{1ex}
    \end{boxedminipage}

    \label{cuon:Email2:Attachments:get_list}
    \index{cuon \textit{(package)}!cuon.Email2 \textit{(module)}!cuon.Email2.Attachments \textit{(class)}!cuon.Email2.Attachments.get\_list \textit{(method)}}

    \vspace{0.5ex}

\hspace{.8\funcindent}\begin{boxedminipage}{\funcwidth}

    \raggedright \textbf{get\_list}(\textit{self})

    \vspace{-1.5ex}

    \rule{\textwidth}{0.5\fboxrule}
\setlength{\parskip}{2ex}
    en: Returns the attachments, as list de: Gibt die Anhaenge als Liste 
    zurueck

\setlength{\parskip}{1ex}
    \end{boxedminipage}


\large{\textbf{\textit{Inherited from object}}}

\begin{quote}
\_\_delattr\_\_(), \_\_format\_\_(), \_\_getattribute\_\_(), \_\_hash\_\_(), \_\_new\_\_(), \_\_reduce\_\_(), \_\_reduce\_ex\_\_(), \_\_repr\_\_(), \_\_setattr\_\_(), \_\_sizeof\_\_(), \_\_str\_\_(), \_\_subclasshook\_\_()
\end{quote}

%%%%%%%%%%%%%%%%%%%%%%%%%%%%%%%%%%%%%%%%%%%%%%%%%%%%%%%%%%%%%%%%%%%%%%%%%%%
%%                              Properties                               %%
%%%%%%%%%%%%%%%%%%%%%%%%%%%%%%%%%%%%%%%%%%%%%%%%%%%%%%%%%%%%%%%%%%%%%%%%%%%

  \subsubsection{Properties}

    \vspace{-1cm}
\hspace{\varindent}\begin{longtable}{|p{\varnamewidth}|p{\vardescrwidth}|l}
\cline{1-2}
\cline{1-2} \centering \textbf{Name} & \centering \textbf{Description}& \\
\cline{1-2}
\endhead\cline{1-2}\multicolumn{3}{r}{\small\textit{continued on next page}}\\\endfoot\cline{1-2}
\endlastfoot\multicolumn{2}{|l|}{\textit{Inherited from object}}\\
\multicolumn{2}{|p{\varwidth}|}{\raggedright \_\_class\_\_}\\
\cline{1-2}
\end{longtable}

    \index{cuon \textit{(package)}!cuon.Email2 \textit{(module)}!cuon.Email2.Attachments \textit{(class)}|)}

%%%%%%%%%%%%%%%%%%%%%%%%%%%%%%%%%%%%%%%%%%%%%%%%%%%%%%%%%%%%%%%%%%%%%%%%%%%
%%                           Class Description                           %%
%%%%%%%%%%%%%%%%%%%%%%%%%%%%%%%%%%%%%%%%%%%%%%%%%%%%%%%%%%%%%%%%%%%%%%%%%%%

    \index{cuon \textit{(package)}!cuon.Email2 \textit{(module)}!cuon.Email2.Recipients \textit{(class)}|(}
\subsection{Class Recipients}

    \label{cuon:Email2:Recipients}
\begin{tabular}{cccccc}
% Line for object, linespec=[False]
\multicolumn{2}{r}{\settowidth{\BCL}{object}\multirow{2}{\BCL}{object}}
&&
  \\\cline{3-3}
  &&\multicolumn{1}{c|}{}
&&
  \\
&&\multicolumn{2}{l}{\textbf{cuon.Email2.Recipients}}
\end{tabular}

\textbf{Known Subclasses:}
cuon.Email2.BCCRecipients,
    cuon.Email2.CCRecipients

en: This object stands for all recipients from the email de: Dieses Objekt 
stellt die Empfaenger einer Email dar


%%%%%%%%%%%%%%%%%%%%%%%%%%%%%%%%%%%%%%%%%%%%%%%%%%%%%%%%%%%%%%%%%%%%%%%%%%%
%%                                Methods                                %%
%%%%%%%%%%%%%%%%%%%%%%%%%%%%%%%%%%%%%%%%%%%%%%%%%%%%%%%%%%%%%%%%%%%%%%%%%%%

  \subsubsection{Methods}

    \vspace{0.5ex}

\hspace{.8\funcindent}\begin{boxedminipage}{\funcwidth}

    \raggedright \textbf{\_\_init\_\_}(\textit{self})

    \vspace{-1.5ex}

    \rule{\textwidth}{0.5\fboxrule}
\setlength{\parskip}{2ex}
    en: Initializes the recipients de: Initialisiert die Empfaenger

\setlength{\parskip}{1ex}
      Overrides: object.\_\_init\_\_

    \end{boxedminipage}

    \label{cuon:Email2:Recipients:add}
    \index{cuon \textit{(package)}!cuon.Email2 \textit{(module)}!cuon.Email2.Recipients \textit{(class)}!cuon.Email2.Recipients.add \textit{(method)}}

    \vspace{0.5ex}

\hspace{.8\funcindent}\begin{boxedminipage}{\funcwidth}

    \raggedright \textbf{add}(\textit{self}, \textit{address}, \textit{caption}={\tt \texttt{'}\texttt{}\texttt{'}})

    \vspace{-1.5ex}

    \rule{\textwidth}{0.5\fboxrule}
\setlength{\parskip}{2ex}
\begin{alltt}

en: Adds a new address to the list of recipients
    address = email address of the recipient
    caption = caption (name) of the recipient
de: Fuegt der Empfaengerliste eine neue Adresse hinzu.
    address = Emailadresse des Empfaengrs
    caption = Bezeichnung (Name) des Empfaengers
\end{alltt}

\setlength{\parskip}{1ex}
    \end{boxedminipage}

    \label{cuon:Email2:Recipients:count}
    \index{cuon \textit{(package)}!cuon.Email2 \textit{(module)}!cuon.Email2.Recipients \textit{(class)}!cuon.Email2.Recipients.count \textit{(method)}}

    \vspace{0.5ex}

\hspace{.8\funcindent}\begin{boxedminipage}{\funcwidth}

    \raggedright \textbf{count}(\textit{self})

    \vspace{-1.5ex}

    \rule{\textwidth}{0.5\fboxrule}
\setlength{\parskip}{2ex}
    en: Returns the number of recipients de: Gibt die Anzahl der Empfaenger
    zurueck

\setlength{\parskip}{1ex}
    \end{boxedminipage}

    \vspace{0.5ex}

\hspace{.8\funcindent}\begin{boxedminipage}{\funcwidth}

    \raggedright \textbf{\_\_repr\_\_}(\textit{self})

    \vspace{-1.5ex}

    \rule{\textwidth}{0.5\fboxrule}
\setlength{\parskip}{2ex}
    en: Returns the list of recipients, as string de: Gibt die 
    Empfengerliste als String zurueck

\setlength{\parskip}{1ex}
      Overrides: object.\_\_repr\_\_

    \end{boxedminipage}

    \label{cuon:Email2:Recipients:get_list}
    \index{cuon \textit{(package)}!cuon.Email2 \textit{(module)}!cuon.Email2.Recipients \textit{(class)}!cuon.Email2.Recipients.get\_list \textit{(method)}}

    \vspace{0.5ex}

\hspace{.8\funcindent}\begin{boxedminipage}{\funcwidth}

    \raggedright \textbf{get\_list}(\textit{self})

    \vspace{-1.5ex}

    \rule{\textwidth}{0.5\fboxrule}
\setlength{\parskip}{2ex}
    en: Returns the list of recipients, as list de: Gibt die 
    Empfaengerliste als Liste zurueck

\setlength{\parskip}{1ex}
    \end{boxedminipage}


\large{\textbf{\textit{Inherited from object}}}

\begin{quote}
\_\_delattr\_\_(), \_\_format\_\_(), \_\_getattribute\_\_(), \_\_hash\_\_(), \_\_new\_\_(), \_\_reduce\_\_(), \_\_reduce\_ex\_\_(), \_\_setattr\_\_(), \_\_sizeof\_\_(), \_\_str\_\_(), \_\_subclasshook\_\_()
\end{quote}

%%%%%%%%%%%%%%%%%%%%%%%%%%%%%%%%%%%%%%%%%%%%%%%%%%%%%%%%%%%%%%%%%%%%%%%%%%%
%%                              Properties                               %%
%%%%%%%%%%%%%%%%%%%%%%%%%%%%%%%%%%%%%%%%%%%%%%%%%%%%%%%%%%%%%%%%%%%%%%%%%%%

  \subsubsection{Properties}

    \vspace{-1cm}
\hspace{\varindent}\begin{longtable}{|p{\varnamewidth}|p{\vardescrwidth}|l}
\cline{1-2}
\cline{1-2} \centering \textbf{Name} & \centering \textbf{Description}& \\
\cline{1-2}
\endhead\cline{1-2}\multicolumn{3}{r}{\small\textit{continued on next page}}\\\endfoot\cline{1-2}
\endlastfoot\multicolumn{2}{|l|}{\textit{Inherited from object}}\\
\multicolumn{2}{|p{\varwidth}|}{\raggedright \_\_class\_\_}\\
\cline{1-2}
\end{longtable}

    \index{cuon \textit{(package)}!cuon.Email2 \textit{(module)}!cuon.Email2.Recipients \textit{(class)}|)}

%%%%%%%%%%%%%%%%%%%%%%%%%%%%%%%%%%%%%%%%%%%%%%%%%%%%%%%%%%%%%%%%%%%%%%%%%%%
%%                           Class Description                           %%
%%%%%%%%%%%%%%%%%%%%%%%%%%%%%%%%%%%%%%%%%%%%%%%%%%%%%%%%%%%%%%%%%%%%%%%%%%%

    \index{cuon \textit{(package)}!cuon.Email2 \textit{(module)}!cuon.Email2.CCRecipients \textit{(class)}|(}
\subsection{Class CCRecipients}

    \label{cuon:Email2:CCRecipients}
\begin{tabular}{cccccccc}
% Line for object, linespec=[False, False]
\multicolumn{2}{r}{\settowidth{\BCL}{object}\multirow{2}{\BCL}{object}}
&&
&&
  \\\cline{3-3}
  &&\multicolumn{1}{c|}{}
&&
&&
  \\
% Line for cuon.Email2.Recipients, linespec=[False]
\multicolumn{4}{r}{\settowidth{\BCL}{cuon.Email2.Recipients}\multirow{2}{\BCL}{cuon.Email2.Recipients}}
&&
  \\\cline{5-5}
  &&&&\multicolumn{1}{c|}{}
&&
  \\
&&&&\multicolumn{2}{l}{\textbf{cuon.Email2.CCRecipients}}
\end{tabular}

en: This object stands for all CC-recipients from the email de: Dieses 
Objekt stellt die CC-Empfaenger einer Email dar


%%%%%%%%%%%%%%%%%%%%%%%%%%%%%%%%%%%%%%%%%%%%%%%%%%%%%%%%%%%%%%%%%%%%%%%%%%%
%%                                Methods                                %%
%%%%%%%%%%%%%%%%%%%%%%%%%%%%%%%%%%%%%%%%%%%%%%%%%%%%%%%%%%%%%%%%%%%%%%%%%%%

  \subsubsection{Methods}

    \vspace{0.5ex}

\hspace{.8\funcindent}\begin{boxedminipage}{\funcwidth}

    \raggedright \textbf{\_\_init\_\_}(\textit{self})

    \vspace{-1.5ex}

    \rule{\textwidth}{0.5\fboxrule}
\setlength{\parskip}{2ex}
    en: Initializes the recipients de: Initialisiert die Empfaenger

\setlength{\parskip}{1ex}
      Overrides: object.\_\_init\_\_

    \end{boxedminipage}

    \vspace{0.5ex}

\hspace{.8\funcindent}\begin{boxedminipage}{\funcwidth}

    \raggedright \textbf{add}(\textit{self}, \textit{address}, \textit{caption}={\tt \texttt{'}\texttt{}\texttt{'}})

    \vspace{-1.5ex}

    \rule{\textwidth}{0.5\fboxrule}
\setlength{\parskip}{2ex}
\begin{alltt}

en: Adds a new address to the list of recipients
    address = email address of the recipient
    caption = caption (name) of the recipient
de: Fuegt der Empfaengerliste eine neue Adresse hinzu.
    address = Emailadresse des Empfaengrs
    caption = Bezeichnung (Name) des Empfaengers
\end{alltt}

\setlength{\parskip}{1ex}
      Overrides: cuon.Email2.Recipients.add

    \end{boxedminipage}

    \vspace{0.5ex}

\hspace{.8\funcindent}\begin{boxedminipage}{\funcwidth}

    \raggedright \textbf{count}(\textit{self})

    \vspace{-1.5ex}

    \rule{\textwidth}{0.5\fboxrule}
\setlength{\parskip}{2ex}
    en: Returns the number of recipients de: Gibt die Anzahl der Empfaenger
    zurueck

\setlength{\parskip}{1ex}
      Overrides: cuon.Email2.Recipients.count

    \end{boxedminipage}

    \vspace{0.5ex}

\hspace{.8\funcindent}\begin{boxedminipage}{\funcwidth}

    \raggedright \textbf{\_\_repr\_\_}(\textit{self})

    \vspace{-1.5ex}

    \rule{\textwidth}{0.5\fboxrule}
\setlength{\parskip}{2ex}
    en: Returns the list of recipients, as string de: Gibt die 
    Empfengerliste als String zurueck

\setlength{\parskip}{1ex}
      Overrides: object.\_\_repr\_\_

    \end{boxedminipage}

    \vspace{0.5ex}

\hspace{.8\funcindent}\begin{boxedminipage}{\funcwidth}

    \raggedright \textbf{get\_list}(\textit{self})

    \vspace{-1.5ex}

    \rule{\textwidth}{0.5\fboxrule}
\setlength{\parskip}{2ex}
    en: Returns the list of recipients, as list de: Gibt die 
    Empfaengerliste als Liste zurueck

\setlength{\parskip}{1ex}
      Overrides: cuon.Email2.Recipients.get\_list

    \end{boxedminipage}


\large{\textbf{\textit{Inherited from object}}}

\begin{quote}
\_\_delattr\_\_(), \_\_format\_\_(), \_\_getattribute\_\_(), \_\_hash\_\_(), \_\_new\_\_(), \_\_reduce\_\_(), \_\_reduce\_ex\_\_(), \_\_setattr\_\_(), \_\_sizeof\_\_(), \_\_str\_\_(), \_\_subclasshook\_\_()
\end{quote}

%%%%%%%%%%%%%%%%%%%%%%%%%%%%%%%%%%%%%%%%%%%%%%%%%%%%%%%%%%%%%%%%%%%%%%%%%%%
%%                              Properties                               %%
%%%%%%%%%%%%%%%%%%%%%%%%%%%%%%%%%%%%%%%%%%%%%%%%%%%%%%%%%%%%%%%%%%%%%%%%%%%

  \subsubsection{Properties}

    \vspace{-1cm}
\hspace{\varindent}\begin{longtable}{|p{\varnamewidth}|p{\vardescrwidth}|l}
\cline{1-2}
\cline{1-2} \centering \textbf{Name} & \centering \textbf{Description}& \\
\cline{1-2}
\endhead\cline{1-2}\multicolumn{3}{r}{\small\textit{continued on next page}}\\\endfoot\cline{1-2}
\endlastfoot\raggedright c\-c\-\_\-r\-e\-c\-i\-p\-i\-e\-n\-t\-s\- & &\\
\cline{1-2}
\multicolumn{2}{|l|}{\textit{Inherited from object}}\\
\multicolumn{2}{|p{\varwidth}|}{\raggedright \_\_class\_\_}\\
\cline{1-2}
\end{longtable}

    \index{cuon \textit{(package)}!cuon.Email2 \textit{(module)}!cuon.Email2.CCRecipients \textit{(class)}|)}

%%%%%%%%%%%%%%%%%%%%%%%%%%%%%%%%%%%%%%%%%%%%%%%%%%%%%%%%%%%%%%%%%%%%%%%%%%%
%%                           Class Description                           %%
%%%%%%%%%%%%%%%%%%%%%%%%%%%%%%%%%%%%%%%%%%%%%%%%%%%%%%%%%%%%%%%%%%%%%%%%%%%

    \index{cuon \textit{(package)}!cuon.Email2 \textit{(module)}!cuon.Email2.BCCRecipients \textit{(class)}|(}
\subsection{Class BCCRecipients}

    \label{cuon:Email2:BCCRecipients}
\begin{tabular}{cccccccc}
% Line for object, linespec=[False, False]
\multicolumn{2}{r}{\settowidth{\BCL}{object}\multirow{2}{\BCL}{object}}
&&
&&
  \\\cline{3-3}
  &&\multicolumn{1}{c|}{}
&&
&&
  \\
% Line for cuon.Email2.Recipients, linespec=[False]
\multicolumn{4}{r}{\settowidth{\BCL}{cuon.Email2.Recipients}\multirow{2}{\BCL}{cuon.Email2.Recipients}}
&&
  \\\cline{5-5}
  &&&&\multicolumn{1}{c|}{}
&&
  \\
&&&&\multicolumn{2}{l}{\textbf{cuon.Email2.BCCRecipients}}
\end{tabular}

en: This object stands for all BCC-recipients from the email de: Dieses 
Objekt stellt die BCC-Empfaenger einer Email dar


%%%%%%%%%%%%%%%%%%%%%%%%%%%%%%%%%%%%%%%%%%%%%%%%%%%%%%%%%%%%%%%%%%%%%%%%%%%
%%                                Methods                                %%
%%%%%%%%%%%%%%%%%%%%%%%%%%%%%%%%%%%%%%%%%%%%%%%%%%%%%%%%%%%%%%%%%%%%%%%%%%%

  \subsubsection{Methods}

    \vspace{0.5ex}

\hspace{.8\funcindent}\begin{boxedminipage}{\funcwidth}

    \raggedright \textbf{\_\_init\_\_}(\textit{self})

    \vspace{-1.5ex}

    \rule{\textwidth}{0.5\fboxrule}
\setlength{\parskip}{2ex}
    en: Initializes the recipients de: Initialisiert die Empfaenger

\setlength{\parskip}{1ex}
      Overrides: object.\_\_init\_\_

    \end{boxedminipage}

    \vspace{0.5ex}

\hspace{.8\funcindent}\begin{boxedminipage}{\funcwidth}

    \raggedright \textbf{add}(\textit{self}, \textit{address}, \textit{caption}={\tt \texttt{'}\texttt{}\texttt{'}})

    \vspace{-1.5ex}

    \rule{\textwidth}{0.5\fboxrule}
\setlength{\parskip}{2ex}
\begin{alltt}

en: Adds a new address to the list of recipients
    address = email address of the recipient
    caption = caption (name) of the recipient
de: Fuegt der Empfaengerliste eine neue Adresse hinzu.
    address = Emailadresse des Empfaengrs
    caption = Bezeichnung (Name) des Empfaengers
\end{alltt}

\setlength{\parskip}{1ex}
      Overrides: cuon.Email2.Recipients.add

    \end{boxedminipage}

    \vspace{0.5ex}

\hspace{.8\funcindent}\begin{boxedminipage}{\funcwidth}

    \raggedright \textbf{count}(\textit{self})

    \vspace{-1.5ex}

    \rule{\textwidth}{0.5\fboxrule}
\setlength{\parskip}{2ex}
    en: Returns the number of recipients de: Gibt die Anzahl der Empfaenger
    zurueck

\setlength{\parskip}{1ex}
      Overrides: cuon.Email2.Recipients.count

    \end{boxedminipage}

    \vspace{0.5ex}

\hspace{.8\funcindent}\begin{boxedminipage}{\funcwidth}

    \raggedright \textbf{\_\_repr\_\_}(\textit{self})

    \vspace{-1.5ex}

    \rule{\textwidth}{0.5\fboxrule}
\setlength{\parskip}{2ex}
    en: Returns the list of recipients, as string de: Gibt die 
    Empfengerliste als String zurueck

\setlength{\parskip}{1ex}
      Overrides: object.\_\_repr\_\_

    \end{boxedminipage}

    \vspace{0.5ex}

\hspace{.8\funcindent}\begin{boxedminipage}{\funcwidth}

    \raggedright \textbf{get\_list}(\textit{self})

    \vspace{-1.5ex}

    \rule{\textwidth}{0.5\fboxrule}
\setlength{\parskip}{2ex}
    en: Returns the list of recipients, as list de: Gibt die 
    Empfaengerliste als Liste zurueck

\setlength{\parskip}{1ex}
      Overrides: cuon.Email2.Recipients.get\_list

    \end{boxedminipage}


\large{\textbf{\textit{Inherited from object}}}

\begin{quote}
\_\_delattr\_\_(), \_\_format\_\_(), \_\_getattribute\_\_(), \_\_hash\_\_(), \_\_new\_\_(), \_\_reduce\_\_(), \_\_reduce\_ex\_\_(), \_\_setattr\_\_(), \_\_sizeof\_\_(), \_\_str\_\_(), \_\_subclasshook\_\_()
\end{quote}

%%%%%%%%%%%%%%%%%%%%%%%%%%%%%%%%%%%%%%%%%%%%%%%%%%%%%%%%%%%%%%%%%%%%%%%%%%%
%%                              Properties                               %%
%%%%%%%%%%%%%%%%%%%%%%%%%%%%%%%%%%%%%%%%%%%%%%%%%%%%%%%%%%%%%%%%%%%%%%%%%%%

  \subsubsection{Properties}

    \vspace{-1cm}
\hspace{\varindent}\begin{longtable}{|p{\varnamewidth}|p{\vardescrwidth}|l}
\cline{1-2}
\cline{1-2} \centering \textbf{Name} & \centering \textbf{Description}& \\
\cline{1-2}
\endhead\cline{1-2}\multicolumn{3}{r}{\small\textit{continued on next page}}\\\endfoot\cline{1-2}
\endlastfoot\raggedright b\-c\-c\-\_\-r\-e\-c\-i\-p\-i\-e\-n\-t\-s\- & &\\
\cline{1-2}
\multicolumn{2}{|l|}{\textit{Inherited from object}}\\
\multicolumn{2}{|p{\varwidth}|}{\raggedright \_\_class\_\_}\\
\cline{1-2}
\end{longtable}

    \index{cuon \textit{(package)}!cuon.Email2 \textit{(module)}!cuon.Email2.BCCRecipients \textit{(class)}|)}

%%%%%%%%%%%%%%%%%%%%%%%%%%%%%%%%%%%%%%%%%%%%%%%%%%%%%%%%%%%%%%%%%%%%%%%%%%%
%%                           Class Description                           %%
%%%%%%%%%%%%%%%%%%%%%%%%%%%%%%%%%%%%%%%%%%%%%%%%%%%%%%%%%%%%%%%%%%%%%%%%%%%

    \index{cuon \textit{(package)}!cuon.Email2 \textit{(module)}!cuon.Email2.Email \textit{(class)}|(}
\subsection{Class Email}

    \label{cuon:Email2:Email}
\begin{tabular}{cccccc}
% Line for object, linespec=[False]
\multicolumn{2}{r}{\settowidth{\BCL}{object}\multirow{2}{\BCL}{object}}
&&
  \\\cline{3-3}
  &&\multicolumn{1}{c|}{}
&&
  \\
&&\multicolumn{2}{l}{\textbf{cuon.Email2.Email}}
\end{tabular}

\begin{alltt}

en: This Object stands for one email, which can sent with
    the method 'send'.
de: Dieses Objekt stellt eine Email dar, welche mit der
    Methode 'send' verschickt werden kann.
\end{alltt}


%%%%%%%%%%%%%%%%%%%%%%%%%%%%%%%%%%%%%%%%%%%%%%%%%%%%%%%%%%%%%%%%%%%%%%%%%%%
%%                                Methods                                %%
%%%%%%%%%%%%%%%%%%%%%%%%%%%%%%%%%%%%%%%%%%%%%%%%%%%%%%%%%%%%%%%%%%%%%%%%%%%

  \subsubsection{Methods}

    \vspace{0.5ex}

\hspace{.8\funcindent}\begin{boxedminipage}{\funcwidth}

    \raggedright \textbf{\_\_init\_\_}(\textit{self}, \textit{from\_address}={\tt \texttt{'}\texttt{}\texttt{'}}, \textit{from\_caption}={\tt \texttt{'}\texttt{}\texttt{'}}, \textit{to\_address}={\tt \texttt{'}\texttt{}\texttt{'}}, \textit{to\_caption}={\tt \texttt{'}\texttt{}\texttt{'}}, \textit{subject}={\tt \texttt{'}\texttt{}\texttt{'}}, \textit{message}={\tt \texttt{'}\texttt{}\texttt{'}}, \textit{smtp\_server}={\tt \texttt{'}\texttt{localhost}\texttt{'}}, \textit{smtp\_user}={\tt \texttt{'}\texttt{}\texttt{'}}, \textit{smtp\_password}={\tt \texttt{'}\texttt{}\texttt{'}}, \textit{attachment\_file}={\tt \texttt{'}\texttt{}\texttt{'}}, \textit{user\_agent}={\tt \texttt{'}\texttt{}\texttt{'}}, \textit{reply\_to\_address}={\tt \texttt{'}\texttt{}\texttt{'}}, \textit{reply\_to\_caption}={\tt \texttt{'}\texttt{}\texttt{'}})

    \vspace{-1.5ex}

    \rule{\textwidth}{0.5\fboxrule}
\setlength{\parskip}{2ex}
\begin{alltt}

en: Initialize the email object
    from\_address     = the email address of the sender
    from\_caption     = the caption (name) of the sender
    to\_address       = the email address of the recipient
    to\_caption       = the caption (name) of the recipient
    subject          = the subject of the email message
    message          = the body text of the email message
    smtp\_server      = the ip-address or the name of the SMTP-server
    smtp\_user        = (optional) Login name for the SMTP-Server
    smtp\_passwort    = (optional) Password for the SMTP-Server
    user\_agent       = (optional) program identification
    reply\_to\_address = (optional) Reply-to email address
    reply\_to\_caption = (optional) Reply-to caption (name)

de: Initialisiert das Email-Objekt
    from\_address     = die Emailadresse des Absenders
    from\_caption     = die Beschrifung (der Name) des Absenders
    to\_address       = die Emailadresse des Empfaengers
    to\_caption       = die Beschriftung (der Name) des Empfaengers
    subject          = der Betreff der Emailnachricht
    message          = der Nachrichtentext
    smtp\_server      = IP-Adresse oder Name des SMTP-Servers.
                       Es kann auch der Port mit ``:`` vom Server getrennt 
                       angehaengt werden. (z.B. ``localhost:25``)
    smtp\_user        = (optional) Benutzername zum Anmelden an den SMTP-Server
    smtp\_passwort    = (optional) das Passwort zum Anmelden an den SMTP-Server
    user\_agent       = (optional) Programm-Identifikation
    reply\_to\_address = (optional) Antwort-an Emailadresse
    reply\_to\_caption = (optional) Antwort-an Beschriftung (Name)
\end{alltt}

\setlength{\parskip}{1ex}
      Overrides: object.\_\_init\_\_

    \end{boxedminipage}

    \label{cuon:Email2:Email:send}
    \index{cuon \textit{(package)}!cuon.Email2 \textit{(module)}!cuon.Email2.Email \textit{(class)}!cuon.Email2.Email.send \textit{(method)}}

    \vspace{0.5ex}

\hspace{.8\funcindent}\begin{boxedminipage}{\funcwidth}

    \raggedright \textbf{send}(\textit{self})

    \vspace{-1.5ex}

    \rule{\textwidth}{0.5\fboxrule}
\setlength{\parskip}{2ex}
\begin{alltt}

de: Sendet die Email an den Empfaenger.
    Wird das Email nur an einen Empfaenger gesendet, dann wird bei
    Erfolg {\textless}True{\textgreater} zurueck gegeben. Wird das Email an mehrere Empfaenger
    gesendet und wurde an mindestens einen der Empfaenger erfolgreich
    ausgeliefert, dann wird ebenfalls {\textless}True{\textgreater} zurueck gegeben.
    
    Wird das Email nur an einen Empfaenger gesendet, dann wird bei
    Misserfolg {\textless}False{\textgreater} zurueck gegeben. Wird das Email an mehrere 
    Empfaenger gesendet und wurde an keinen der Empfaenger erfolgreich
    ausgeliefert, dann wird {\textless}False{\textgreater} zurueck gegeben.
\end{alltt}

\setlength{\parskip}{1ex}
    \end{boxedminipage}


\large{\textbf{\textit{Inherited from object}}}

\begin{quote}
\_\_delattr\_\_(), \_\_format\_\_(), \_\_getattribute\_\_(), \_\_hash\_\_(), \_\_new\_\_(), \_\_reduce\_\_(), \_\_reduce\_ex\_\_(), \_\_repr\_\_(), \_\_setattr\_\_(), \_\_sizeof\_\_(), \_\_str\_\_(), \_\_subclasshook\_\_()
\end{quote}

%%%%%%%%%%%%%%%%%%%%%%%%%%%%%%%%%%%%%%%%%%%%%%%%%%%%%%%%%%%%%%%%%%%%%%%%%%%
%%                              Properties                               %%
%%%%%%%%%%%%%%%%%%%%%%%%%%%%%%%%%%%%%%%%%%%%%%%%%%%%%%%%%%%%%%%%%%%%%%%%%%%

  \subsubsection{Properties}

    \vspace{-1cm}
\hspace{\varindent}\begin{longtable}{|p{\varnamewidth}|p{\vardescrwidth}|l}
\cline{1-2}
\cline{1-2} \centering \textbf{Name} & \centering \textbf{Description}& \\
\cline{1-2}
\endhead\cline{1-2}\multicolumn{3}{r}{\small\textit{continued on next page}}\\\endfoot\cline{1-2}
\endlastfoot\raggedright a\-t\-t\-a\-c\-h\-m\-e\-n\-t\-s\- & &\\
\cline{1-2}
\raggedright b\-c\-c\-\_\-r\-e\-c\-i\-p\-i\-e\-n\-t\-s\- & &\\
\cline{1-2}
\raggedright c\-c\-\_\-r\-e\-c\-i\-p\-i\-e\-n\-t\-s\- & &\\
\cline{1-2}
\raggedright c\-o\-n\-t\-e\-n\-t\-\_\-c\-h\-a\-r\-s\-e\-t\- & &\\
\cline{1-2}
\raggedright c\-o\-n\-t\-e\-n\-t\-\_\-s\-u\-b\-t\-y\-p\-e\- & &\\
\cline{1-2}
\raggedright f\-r\-o\-m\-\_\-a\-d\-d\-r\-e\-s\-s\- & &\\
\cline{1-2}
\raggedright f\-r\-o\-m\-\_\-c\-a\-p\-t\-i\-o\-n\- & &\\
\cline{1-2}
\raggedright h\-e\-a\-d\-e\-r\-\_\-c\-h\-a\-r\-s\-e\-t\- & &\\
\cline{1-2}
\raggedright m\-e\-s\-s\-a\-g\-e\- & &\\
\cline{1-2}
\raggedright r\-e\-c\-i\-p\-i\-e\-n\-t\-s\- & &\\
\cline{1-2}
\raggedright r\-e\-p\-l\-y\-\_\-t\-o\-\_\-a\-d\-d\-r\-e\-s\-s\- & &\\
\cline{1-2}
\raggedright r\-e\-p\-l\-y\-\_\-t\-o\-\_\-c\-a\-p\-t\-i\-o\-n\- & &\\
\cline{1-2}
\raggedright s\-m\-t\-p\-\_\-c\-r\-y\-p\-t\- & &\\
\cline{1-2}
\raggedright s\-m\-t\-p\-\_\-p\-a\-s\-s\-w\-o\-r\-d\- & &\\
\cline{1-2}
\raggedright s\-m\-t\-p\-\_\-s\-e\-r\-v\-e\-r\- & &\\
\cline{1-2}
\raggedright s\-m\-t\-p\-\_\-u\-s\-e\-r\- & &\\
\cline{1-2}
\raggedright s\-t\-a\-t\-u\-s\-d\-i\-c\-t\- & &\\
\cline{1-2}
\raggedright s\-u\-b\-j\-e\-c\-t\- & &\\
\cline{1-2}
\raggedright u\-s\-e\-r\-\_\-a\-g\-e\-n\-t\- & &\\
\cline{1-2}
\multicolumn{2}{|l|}{\textit{Inherited from object}}\\
\multicolumn{2}{|p{\varwidth}|}{\raggedright \_\_class\_\_}\\
\cline{1-2}
\end{longtable}

    \index{cuon \textit{(package)}!cuon.Email2 \textit{(module)}!cuon.Email2.Email \textit{(class)}|)}
    \index{cuon \textit{(package)}!cuon.Email2 \textit{(module)}|)}
