%
% API Documentation for API Documentation
% Module cuon.Databases.xbase
%
% Generated by epydoc 3.0.1
% [Fri Feb 11 22:17:50 2011]
%

%%%%%%%%%%%%%%%%%%%%%%%%%%%%%%%%%%%%%%%%%%%%%%%%%%%%%%%%%%%%%%%%%%%%%%%%%%%
%%                          Module Description                           %%
%%%%%%%%%%%%%%%%%%%%%%%%%%%%%%%%%%%%%%%%%%%%%%%%%%%%%%%%%%%%%%%%%%%%%%%%%%%

    \index{cuon \textit{(package)}!cuon.Databases \textit{(package)}!cuon.Databases.xbase \textit{(module)}|(}
\section{Module cuon.Databases.xbase}

    \label{cuon:Databases:xbase}

%%%%%%%%%%%%%%%%%%%%%%%%%%%%%%%%%%%%%%%%%%%%%%%%%%%%%%%%%%%%%%%%%%%%%%%%%%%
%%                               Functions                               %%
%%%%%%%%%%%%%%%%%%%%%%%%%%%%%%%%%%%%%%%%%%%%%%%%%%%%%%%%%%%%%%%%%%%%%%%%%%%

  \subsection{Functions}

    \label{cuon:Databases:xbase:dbfreader}
    \index{cuon \textit{(package)}!cuon.Databases \textit{(package)}!cuon.Databases.xbase \textit{(module)}!cuon.Databases.xbase.dbfreader \textit{(function)}}

    \vspace{0.5ex}

\hspace{.8\funcindent}\begin{boxedminipage}{\funcwidth}

    \raggedright \textbf{dbfreader}(\textit{f})

    \vspace{-1.5ex}

    \rule{\textwidth}{0.5\fboxrule}
\setlength{\parskip}{2ex}
    Returns an iterator over records in a Xbase DBF file.

    The first row returned contains the field names. The second row 
    contains field specs: (type, size, decimal places). Subsequent rows 
    contain the data records. If a record is marked as deleted, it is 
    skipped.

    File should be opened for binary reads.

\setlength{\parskip}{1ex}
    \end{boxedminipage}

    \label{cuon:Databases:xbase:dbfwriter}
    \index{cuon \textit{(package)}!cuon.Databases \textit{(package)}!cuon.Databases.xbase \textit{(module)}!cuon.Databases.xbase.dbfwriter \textit{(function)}}

    \vspace{0.5ex}

\hspace{.8\funcindent}\begin{boxedminipage}{\funcwidth}

    \raggedright \textbf{dbfwriter}(\textit{f}, \textit{fieldnames}, \textit{fieldspecs}, \textit{records})

    \vspace{-1.5ex}

    \rule{\textwidth}{0.5\fboxrule}
\setlength{\parskip}{2ex}
\begin{alltt}
Return a string suitable for writing directly to a binary dbf file.

File f should be open for writing in a binary mode.

Fieldnames should be no longer than ten characters and not include {\textbackslash}.
Fieldspecs are in the form (type, size, deci) where
    type is one of:
        C for ascii character data
        M for ascii character memo data (real memo fields not supported)
        D for datetime objects
        N for ints or decimal objects
        L for logical values 'T', 'F', or '?'
    size is the field width
    deci is the number of decimal places in the provided decimal object
Records can be an iterable over the records (sequences of field values).
\end{alltt}

\setlength{\parskip}{1ex}
    \end{boxedminipage}


%%%%%%%%%%%%%%%%%%%%%%%%%%%%%%%%%%%%%%%%%%%%%%%%%%%%%%%%%%%%%%%%%%%%%%%%%%%
%%                               Variables                               %%
%%%%%%%%%%%%%%%%%%%%%%%%%%%%%%%%%%%%%%%%%%%%%%%%%%%%%%%%%%%%%%%%%%%%%%%%%%%

  \subsection{Variables}

    \vspace{-1cm}
\hspace{\varindent}\begin{longtable}{|p{\varnamewidth}|p{\vardescrwidth}|l}
\cline{1-2}
\cline{1-2} \centering \textbf{Name} & \centering \textbf{Description}& \\
\cline{1-2}
\endhead\cline{1-2}\multicolumn{3}{r}{\small\textit{continued on next page}}\\\endfoot\cline{1-2}
\endlastfoot\raggedright \_\-\_\-p\-a\-c\-k\-a\-g\-e\-\_\-\_\- & \raggedright \textbf{Value:} 
{\tt \texttt{'}\texttt{cuon.Databases}\texttt{'}}&\\
\cline{1-2}
\end{longtable}

    \index{cuon \textit{(package)}!cuon.Databases \textit{(package)}!cuon.Databases.xbase \textit{(module)}|)}
